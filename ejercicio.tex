\documentclass{article}

\usepackage[utf8]{inputenc}
\usepackage[spanish]{babel}
\usepackage{amsmath}
\usepackage{gensymb}
\begin{document}

\begin{enumerate}
\item \textbf{Números impares}
  
Resuelve la siguiente ecuación en diferencias
$$x_{n+2}=2x_{n+1}-x_{n}^{(*)}$$
donde
$$x_{o}=-1,x_{1}=1$$
\textbf{Solución:}

Podriamos intentar resolver el problema planteando la ecuación resolvente.
$$r^{2}=2r-1$$
igualando a cero
$$r^{2}-2r+1=0$$
Sin embargo, al resolver la ecuación de segundo grado nos damos cuenta de que solo tiene una solución $r=1$, por lo cual este método no nos lleva a una solución. Otra opción viable es desarrollar la suseción.
$$x_{2}=2x_{1}-x_{0}=2(1)-(-1)=3$$
$$x_{3}=2x_{2}-x_{1}=2(3)-(1)=5$$
$$x_{4}=2x_{3}-x_{2}=2(5)-(3)=7$$
$$\vdots$$
Podemos observar que la suseción parecen ser los números impares, por lo cual, una posible solución es $x_{n}=2n-1$, lo cual podermos comprobar al sustitur esta  solucón en $(*)$.
$$2x_{n+1}-x_{n}=2(2(n+1)-1)-(2n-1)=2n+3=2(n+2)-1=x_{n+2}$$

\item \textbf{Problema de enfriamiento}

Una taza de café tiene una temperatura incial de $165F^{\circ}$, pero se enfría a $155F^{\circ}$ en un minuto en una habitación con temperatuda de $70F^{\circ}$. Sea $T_{n}$ la temperatura del café después de $n$ minutos, encuentra una formula para $T_{n}$ y grafícala.

\textbf{Solución:}

Por la ley del enfriamiento se sabe que la diferencia entre la temperatura final y la temperatura inicial de un objeto que se enfría es proporcional a la diferencia entre la temperatura inicial y la temperatura ambiental $T_{a}$, e.i. existe una $k$ en los reales, tal que
$$T_{n}-T_{n-1}=k
\left(
  T_{a}-T_{n-1}
\right)$$
Despejando a $k$ y sustituyento los datos en la ecuación se obtiene que
$$k=\frac{T_{n}-T_{n-1}}{ T_{a}-T_{n-1}}=\frac{165-155}{70-165}=-\frac{10}{95}=-\frac{2}{19}$$
De tal manera que ahora podemos plantear una ecuacion en diferencias a partir de
$$T_{n}-T_{n-1}=\frac{2}{19}
\left(
  70-T_{n-1}
\right)$$
$$\Rightarrow T_{n}=\frac{17}{19} T_{n-1}+\frac{140}{19}$$
\end{enumerate}
\end{document}
