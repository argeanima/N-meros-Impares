\documentclass{article}

\usepackage[utf8]{inputenc}
\usepackage[spanish]{babel}
\usepackage{amsmath}

\begin{document}
Resuelve la siguiente ecuación en diferencias
$$x_{n+2}=2x_{n+1}-x_{n}^{(*)}$$

donde

$$x_{o}=-1,x_{1}=1$$

\textbf{Solución:}

Podriamos intentar resolver el problema planteando la ecuación resolvente.
$$r^{2}=2r-1$$
igualando a cero
$$r^{2}-2r+1=0$$
Sin embargo, al resolver la ecuación de segundo grado nos damos cuenta de que solo tiene una solución $r=1$, por lo cual este método no nos lleva a una solución. Otra opción viable es desarrollar la suseción.
$$x_{2}=2x_{1}-x_{0}=2(1)-(-1)=3$$
$$x_{3}=2x_{2}-x_{1}=2(3)-(1)=5$$
$$x_{4}=2x_{3}-x_{2}=2(5)-(3)=7$$
$$\vdots$$
Podemos observar que la suseción parecen ser los números impares, por lo cual, una posible solución es $x_{n}=2n-1$, lo cual podermos comprobar al sustitur esta  solucón en $(*)$.
$$2x_{n+1}-x_{n}=2(2(n+1)-1)-(2n-1)=2n+3=2(n+2)-1=x_{n+2}$$

\end{document}
